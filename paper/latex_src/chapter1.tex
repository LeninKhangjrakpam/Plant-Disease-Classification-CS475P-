
\chapter{Introduction}
\pagenumbering{arabic}

\section{Background}
Machine Learning is a field of Artificial Intelligence that focuses on developing algorithms and models, by learning from patterns in a dataset and using it to make predictions on new data. These algorithms can be very useful for identifying patterns and relationships in a given database, which can be invaluable to extract insights or make predictions without being explicitly programmed.\par \vspace{1em}
Machine Learning can be classified into various categories; supervised learning, unsupervised learning, and reinforcement learning being the most common ones. Supervised Learning Algorithms learn from labelled data, making predictions based on the given data. Unsupervised Learning Algorithms learn from unlabelled data, where the algorithm identifies patterns and structures within the data. Reinforcement Learning Algorithms learns through interacting with an environment, and aiming to maximize cumulative rewards.\par \vspace{1em}
Convolutional Neural Networks or CNNs are a subset of deep learning models especially useful for tasks involving processing of visual data. Its architecture is particularly inspired by the connectivity patterns of the human brain, especially the visual cortex. Its primary goal is to automatically learn the hierarchical representations of features directly from raw data.\par \vspace{1em} \newpage
CNNs use specialized architectures that incorporate convolutional layers, pooling layers, and fully connected layers. Convolutional layers apply convolution operations to the input image, using learnable filters to extract features such as edges, textures, or shapes. Pooling layers then reduce the spatial dimensions of the feature maps generated by the convolutional layers, helping to decrease computational complexity and control overfitting. Finally, fully connected layers use the extracted features to classify the input image into various categories.\par \vspace{1em}

\section{Problem Statement}

Accurate and efficient detection of plant diseases is critical for ensuring food security and agricultural productivity. Traditional methods of plant disease identification often rely on expert knowledge and manual inspection, which can be time-consuming, labor-intensive, and prone to human error. With the advent of deep learning and advanced machine learning techniques, there is a growing potential to automate and enhance the accuracy of disease classification using image-based analysis.\par
The Plant Village dataset provides a robust foundation for developing machine learning models aimed at identifying various plant diseases from images. However, the challenge lies in selecting and optimizing the appropriate model architectures to achieve high classification accuracy. While Convolutional Neural Networks (CNNs) have shown great promise in image recognition tasks, there is a need to systematically compare their performance with traditional classifiers and evaluate the benefits of using transfer learning and fine-tuning techniques with pre-trained models.\par

This study aims to address the following key questions:
\begin{enumerate}
    \item How do pre-trained models like VGG16, Xception, and MobileNetV2 perform in the context of plant disease classification when fine-tuned on the Plant Village dataset?
    \item Can traditional classifiers such as SVM, KNN, Random Forest, and XGBoost, when used with features extracted from pre-trained models, achieve comparable performance to CNN-based models?
    \item What is the effectiveness of a custom-designed CNN model in classifying plant diseases compared to fine-tuned pre-trained models?
    \item How can we interpret and visualize the features learned by CNN models to gain insights into their decision-making processes?
\end{enumerate}
\par

By addressing these questions, this research aims to identify the most effective machine learning approaches for plant disease classification and provide a deeper understanding of the underlying mechanisms of CNN models through visualization techniques. \par



\section{Scope of Study}
The scope of this study encompasses a comprehensive evaluation of various image classification machine learning models applied to the Plant Village dataset, specifically focusing on the detection and classification of plant diseases. The key components of this study are as follows:
\begin{enumerate}
    \item \textbf{Dataset Utilization:} \\
    The Plant Village dataset, consisting of images categorized into 11 different plant disease classes, is used to train and evaluate the models. This dataset provides a diverse and challenging testbed for assessing model performance in real-world scenarios.
    \item \textbf{Model Selection and Training:} 
    \begin{itemize}
        \item  Three pre-trained models, VGG16, Xception, and MobileNetV2, are employed as base models for transfer learning and fine-tuning. These models are chosen for their proven efficacy in image classification tasks and their ability to leverage pre-learned features from large datasets like ImageNet.
        \item In addition to these neural network models, traditional machine learning classifiers such as SVM, KNN, Random Forest, and XGBoost are trained using features extracted from the fine-tuned MobileNetV2 model. This provides a comparative analysis of deep learning models against conventional machine learning approaches.
    \end{itemize}
    \item \textbf{Model Development: }\\
    A custom Convolutional Neural Network (CNN) with 10 layers is designed and trained from scratch, inspired by the VGG architecture. This model serves as a baseline to compare the performance of transfer learning models and to understand the effectiveness of custom architectures on the dataset.
    \item \textbf{Model Interpretation and Analysis:} \\
    Various techniques are employed to interpret and analyze the models, including:
    \begin{itemize}
        \item Visualization of the first layer filters and feature maps at different layers.
        \item Comparative analysis of feature maps using normal and masked images.
        \item Feature space visualization using PCA and t-SNE before and after fine-tuning.
        \item Nearest neighbor search using cosine similarity to evaluate the feature representations.
        \item Saliency Map Generation
    \end{itemize}
    \item \textbf{Deployment:} \\
    The best-performing model is deployed using Flask to serve as an API for real-time image classification. The frontend of the application is developed using React Native, enabling users to submit images and receive predictions.
    \item \textbf{Performance Evaluation:} \\
    The models are evaluated based on their accuracy, confusion matrices, and interpretability. This includes a thorough comparison of the classification performance of neural network models and traditional classifiers.
\end{enumerate}


This study aims to provide a detailed comparative analysis of different machine learning models for plant disease classification, highlighting the strengths and limitations of each approach. By leveraging transfer learning, custom model development, and comprehensive model interpretation techniques, this study contributes valuable insights into the application of machine learning in agricultural disease detection.\par\vspace{1em}
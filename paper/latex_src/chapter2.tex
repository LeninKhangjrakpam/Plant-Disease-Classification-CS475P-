\chapter{Literature Review}

In this chapter, we list all the papers, that we used to to perform our research work \par\vspace{1em}

\section{Literature Review}

\begin{enumerate}
    \item Konstantinos P. Ferentinos ( February 2018) \\
        \textbf{Deep learning models for plant disease detection and diagnosis} \\
        \href{https://www.sciencedirect.com/science/article/abs/pii/S0168169917311742}{https://www.sciencedirect.com/science/article/abs/pii/S0168169917311742} \\
         The paper develops neural network models to perform plant disease detection and diagnosis using simple leaves image of healthy and diseased leaves. Training used an open database of 87,848 images, containing 25 different plants, in a set of 58 different classes.Several model architectures were trained, with the best performance reaching a 99.53\% success rate in reaching the correct diagnosis
    
    \item Sharada Prasanna Mohanty, David Hughes, Marcel Salathé (2016) \\
        \textbf{Using Deep Learning for Image-Based Plant Disease Detection} \\
        \href{https://arxiv.org/abs/1604.03169}{https://arxiv.org/abs/1604.03169} \\
        The paper presents an innovative approach to addressing the significant issue of crop disease diagnosis by leveraging deep learning technology. The authors successfully trained a deep convolutional neural network on a large dataset of plant leaf images, achieving a high accuracy of 99.35\% in controlled conditions. However, the model's performance dropped to 31.4\% when tested with images from different sources, highlighting the need for more diverse training data. This study demonstrates the importance of improving data diversity to enhance model generalizability for real-world applications.
    
    \item Francois Chollet (2017) \\
        \textbf{Xception: Deep Learning with Depthwise Separable Convolutions} \\
        \href{https://doi.org/XXXXXX}{https://doi.org/XXXXXX} \\
        The paper introduces an interpretation of Inception modules in CNN as an intermediate step between regular convolution and depthwise separable convolution (a depthwise convolution followed by a pointwise convolution). By understanding depthwise separable convolution as an Inception module with a maximally large number of towers, the authors propose a new CNN architecture, Xception, which replaces Inception modules with depthwise separable convolutions.  \\
    
    \item Muhammad Hammad Saleem ,Johan Potgieter and Khalid Mahmood Arif (31 October 2019) \\
        \textbf{Plant Disease Detection and Classification by Deep Learning} \\
        \href{https://www.mdpi.com/2223-7747/8/11/468}{https://www.mdpi.com/2223-7747/8/11/468} \\
        This paper provides a comprehensive explanation of DL models used to visualize various plant diseases. In addition, some research gaps are identified from which to obtain greater transparency for detecting diseases in plants, even before their symptoms appear clearly.\\

    \item Muhammad Shoaib Shaker El-sappagh,Babar Shah,Akhtar Ali,Asad Ullah,Fayadh Alenezi,Tsanko Gechev,Tariq Hussain and Farman Ali (2023) \\
        \textbf{An advanced deep learning models-based plant disease detection: A review of recent research } \\
        \href{https://www.frontiersin.org/journals/plant-science/articles/10.3389/fpls.2023.1158933/full}{https://www.frontiersin.org/journals/plant-science/articles/10.3389/fpls.2023.1158933/full} \\
        This paper addresses the challenges and limitations associated with using ML and DL for plant disease identification, such as issues with data availability, imaging quality, and the differentiation between healthy and diseased plants. The research provides valuable insights for plant disease detection researchers, practitioners, and industry professionals by offering solutions to these challenges and limitations, providing a comprehensive understanding of the current state of research in this field, highlighting the benefits and limitations of these methods, and proposing potential solutions to overcome the challenges of their implementation.\\

    \item Edna Chebet Too , Li Yujian , Sam Njuki and Liu Yingchun(June 2019) \\
        \textbf{A comparative study of fine-tuning deep learning models for plant disease identification} \\
        \href{https://www.sciencedirect.com/science/article/abs/pii/S0168169917313303}{https://www.sciencedirect.com/science/article/abs/pii/S0168169917313303} \\
        This paper is a comparative study of state-of-the-art deep learning for plants disease detection using images of leaves. The results show that deeper models are not only accurate but have fewer number of parameters. DenseNet model also perform better than other models studied with no signs of overfitting and performance deterioration.\\

    \item Murk Chohan, Adil Khan, Rozina Chohan, Saif Hassan Katpar and Muhammad Saleem Mahar (May 2020) \\
        \textbf{Plant Disease Detection using Deep Learning} \\
        \href{https://www.researchgate.net/profile/Saif-Katper/publication/341025012_Plant_Disease_Detection_using_Deep_Learning/links/5eabf714299bf18b958a94a8/Plant-Disease-Detection-using-Deep-Learning.pdf}
        {https://www.researchgate.net/profile/Saif-Katper/publication/341025012 \_Plant \_Disease\_Detection\_using\_Deep\_Learning/links/5eabf714299bf18b958a94a8/Plant-Disease-Detection-using-Deep-Learning.pdf} \\
        This paper proposes a deep learning based model named plant disease detector. The model is able to detect several diseases from plants using pictures of their leaves. Plant disease detection model is developed using neural network. First of all augmentation is applied on dataset to increase the sample size. Later Convolution Neural Network (CNN) is used with multiple convolution and pooling layers. PlantVillage dataset is used to train the model. After training the model, it is tested properly to validate the results.

    \item Faye Mohameth, Chen Bingcai and Kane Amath Sada (June 2020) \\
        \textbf{Plant Disease Detection with Deep Learning and Feature Extraction Using Plant Village} \\
        \href{https://www.scirp.org/journal/paperinformation?paperid=100958}{https://www.scirp.org/journal/paperinformation?paperid=100958} \\
        This paper evaluates CNN’s architectures applying transfer learning and deep feature extraction. All the features obtained will also be classified by SVM and KNN. Our work is feasible by the use of the open source Plant Village Dataset. The result obtained shows that SVM is the best classifier for leaf’s diseases detection.
    
    \item Rasim Alguliyev, Yadigar Imamverdiyev, Lyudmila Sukhostat and Ruslan Bayramov  ( 01 September 2021) \\
        \textbf{Plant disease detection based on a deep model} \\
        \href{https://link.springer.com/article/10.1007/s00500-021-06176-4}{https://link.springer.com/article/10.1007/s00500-021-06176-4} \\
        This paper proposes an accurate approach to identify plant leaf diseases based on the deep convolutional neural network and gated recurrent units. The proposed model is trained to identify common plant leaf diseases of 14 species. The experimental results are compared to other well-known models. This study shows that the proposed model based on deep learning provides the best solution in the diagnosis of plant diseases with high accuracy, and that the gated recurrent unit neural network considered as a classifier can improve the accuracy of the convolutional neural network model. 

    \item Shima Ramesh, Ramachandra Hebbar,Niveditha M.,Pooja R., Prasad Bhat N., Shashank N., Vinod P.V. ( 2018) \\
        \textbf{Plant Disease Detection Using Machine Learning} \\
        \href{https://ieeexplore.ieee.org/abstract/document/8437085}{https://ieeexplore.ieee.org/abstract/document/8437085} \\
        This paper makes use of Random Forest in identifying between healthy and diseased leaf from the data sets created. Our proposed paper includes various phases of implementation namely dataset creation, feature extraction, training the classifier and classification. The created datasets of diseased and healthy leaves are collectively trained under Random Forest to classify the diseased and healthy images.
        
    \item \textbf{Laha Ale, Alaa Sheta, Longzhuang Li, Ye Wang, Ning Zhang} (2019 ) \\
        \textbf{Deep Learning Based Plant Disease Detection for Smart Agriculture} \\
        \href{https://ieeexplore.ieee.org/abstract/document/9024439}{https://ieeexplore.ieee.org/abstract/document/9024439} \\
        This paper proposes a Densely Connected Convolutional Networks (DenseNet) based transfer learning method to detect the plant diseases, which expects to run on edge servers with augmented computing resources. It then proposes a lightweight Deep Neural Networks (DNN) approach that can run on Internet of Things (IoT) devices with constrained resources.
    
\end{enumerate}
